\documentclass[12pt]{article}
\usepackage[utf8]{inputenc}
\usepackage[T1]{fontenc}
\usepackage{amsmath, amssymb}
\usepackage{booktabs}
\usepackage{graphicx}
\usepackage{pgfplots}
\pgfplotsset{compat=1.18,width=10cm}
\usepackage{geometry}
\geometry{
 a4paper,
 total={170mm,257mm},
 left=20mm,
 top=20mm,
 }
\begin{document}
\section{\underline{EMF, Internal resistance, Terminal voltage:}}
\Large{\begin{enumerate}
\item The internal resistance of a 2.1 V cell which gives a current of 0.2 A through a resistance of 10 \(\Omega\) is\hfill{\small PW}\\ \quad (A) 0.2 \(\Omega\)\quad\quad (B) 0.5 \(\Omega\) \quad (C) 0.8 \(\Omega\)\quad\quad (D) 1.0 \(\Omega \)\\
\item A current of 2 A flows through a 2 \(\Omega\) resistor when connected across a battery. The same battery supplies a current of 0.5 A when connected across a 9 \( \Omega\) resistor. The internal resistance of the battery is \hfill{\small2011}\\\quad(A) 0.5 \(\Omega\)\quad (B) \(\frac{1}{3}\) \(\Omega\)\quad (C) \(\frac{1}{4}\) \(\Omega\)\quad (D) 1 $\Omega$\quad\\
\item A set of 'n' equal resistors, of value 'R' each are connected in series to a battery of emf 'E' and internal resistance 'R'. The current drawn is I. Now, the 'n' resistors are connected in parallel to the same battery. Then the current drawn from battery becomes 10 I. The value of 'n' is\\
		 (A) 10\quad (B) 11\quad (C) 20\quad (D) 9\quad
\item A car battery of emf 12 V and internal resistance 5 \(\times\) \(10^{-2}\) \(\Omega\), receives a current of 60 amp, from external source, then terminal potential difference of battery is\\
(A) 12 V\quad (B) 9 V\quad
(C) 15 V\quad (D) 20 V\quad
\item For a cell terminal potential difference is 2.2 V
when circuit is open and reduces to 1.8 V
when cell is connected to a resistance of
R = 5 \(\Omega\). Determine internal resistance of
cell(r) \hfill(2002)\\
(A) \(\frac{10}{9}\) \(\Omega\)\quad (B) \(\frac{9}{10}\) \(\Omega\)\quad (C) \(\frac{11}{9}\) \(\Omega\)\quad(D) \(\frac{5}{9}\) \(\Omega\)
\item A student measures the terminal potential
difference (V) of a cell (of e.m.f. \(\varepsilon\) and internal
resistance r) as a function of the current (I)
flowing through it. The slope, and intercept, of
the graph between V and I, then, respectively,
equal\\
(A) -r and \(\varepsilon\)\quad (B) r and -\(\varepsilon\)\quad (C) -\(\varepsilon\) and r\quad (D) \(\varepsilon\) and -r
\item A cell having an emf $\varepsilon$ and internal resistance
r is connected across a variable external
resistance R. As the resistance R is increased,
the plot of potential difference V across R is
given by\\
(A)\includegraphics[]{a.jpg}
(B)\includegraphics[]{b.jpg}\\
(C)\includegraphics[]{c.jpg} 
(D)\includegraphics[]{d.jpg}

\end{enumerate}
 }

\begin{align*}
\end{align*}
\begin{equation*}
\end{equation*}



\end{document}